 \subsection{Known substitutions} One of the table columns is ``substitutions''
- other amino acid types seen at the same position in the alignment. These amino acid types may be interchangeable
at that position in the protein, so if one wants to affect the protein by a point mutation, they should be avoided.
For example if the substitutions are ``RVK'' and the  original protein has an R at that position, it is advisable to
try anything, but RVK. Conversely, when looking for substitutions which will {\emph not}  affect the protein,
one may try replacing, R with K, or (perhaps more surprisingly), with V. The percentage of times the substitution appears in the
alignment is given in the immediately following bracket. No percentage is given in the cases when it is smaller than 1\%.
This is meant to be a rough guide - due to rounding errors these percentages often do not add up to 100\%.

\subsection{Surface} To detect candidates for novel functional interfaces, first we look for residues
that are solvent accessible (according to DSSP program) by at least $10\AA^2$, 
which is roughly the area  needed for one water molecule to come in the contact with the residue.
 Furthermore, we require that these residues form a ``cluster'' of residues which have 
 neighbor within  $5\AA$ from any of their heavy atoms.

 Note, however,  that, if our picture of protein evolution is
correct, the neighboring residues which  \emph {are not} surface accessible might be equally important
in maintaining the interaction specificity - they should not be automatically dropped from consideration when
choosing the set for mutagenesis. (Especially if they form  a cluster with the surface residues.)

\subsection{Number of contacts} Another column worth noting is denoted ``noc/bb''; it tells the number of contacts heavy atoms
of the residue in question make across the interface, as well as  how many of them are realized through
the backbone atoms (if all or most contacts are through the backbone, mutation presumably won't have strong impact).
Two heavy atoms are considered to be ``in contact'' if their centers are closer than $5\AA$.

\subsection{Annotation} If the residue annotation is available (either from the pdb file or from other sources), another column,
with the header ``annotation'' appears. Annotations carried over from PDB are the following: 
site (indicating existence of related  site record in PDB ), S-S (disulfide bond forming residue),  
hb (hydrogen bond forming residue, jb (james bond forming residue), and sb (for salt bridge
forming residue).

{\subsection{Mutation suggestions} \label{mutnotes}
 Mutation suggestions are completely heuristic and based on complementarity with the substitutions found
in the alignment. Note that they are meant to be {\bf disruptive} to the interaction of the protein with its ligand.
The attempt is made to complement the following properties: small $[AVGSTC]$, medium $[
LPNQDEMIK]$, large $[WFYHR]$, 
hydrophobic $[LPVAMWFI]$, polar $[GTCY]$; positively $[KHR]$,
 or negatively $[DE]$ charged, aromatic $[WFYH]$, long aliphatic chain $[EKRQM]$, 
 OH-group possession $[SDETY]$, and NH2 group possession $[NQRK]$.
The suggestions are listed according to how different they appear to be from the original amino acid, 
and they are grouped in round brackets   
if they appear equally  disruptive. From left to right, each bracketed group of amino acid types
resembles more strongly the original (i.e. is, presumably, less disruptive)
These suggestions are tentative - they might prove disruptive to the fold rather than to the interaction.
Many researcher will choose, however, the straightforward alanine mutations, especially in the
beginning stages of their investigation.
