\section{Notes on using trace results}

 {\bf Coverage.} The trace  results are commonly expressed in terms of coverage: the residue is
important if its ``coverage'' is small - that is if it belongs to some small top percentage of residues [100\% is
 all of the residues in a chain], according to trace.
The ET results are presented  in form of a table, listing top percentage of residues,
 sorted by the strength of the evolutionary pressure(as we infer it).
(The smaller the coverage, the stronger the pressure on the residue.) Starting from the top of that list,
mutating a couple of residues should affect the protein somehow -  how exactly, we should hopefully learn from
the experiment.

 {\bf Known substitutions.} One of the table columns is ``substitutions''
- other amino acid types seen at the same position in the alignment. These amino acid types may be interchangeable
at that position in the protein, so if one wants to affect the protein by a point mutation, they should be avoided.
For example if the substitutions are ``RVK'' and my original protein has an R at that position, it is advisable to
try anything, but RVK. To the contrary, if I am looking for substitutions which will {\emph not}  affect the protein,
I may try replacing, R with K, or more surprisingly, with V. The percentage of times the substitution appears in the
alignment if given in the immediately following bracket. No percentage is given in the cases when it is smaller than 1.
This is meant ot be a rough guide - due to rounding errors these percentages often do not add up to 100.

{\bf Annotation.} If the residue annotation is available (either from the Uniprot [indicated by 
``UP/SP'', for UniProt/SwissProt] file or from other sources), another column,
with the header ``annotation'' appears.

{\bf Mutation suggestions.} \label{mutnotes}
Mutation suggestions are completely heuristic and based on complementarity with the substitutions found
in the alignment. Note that they are meant to be {\bf disruptive} to the interaction of the protein with its ligand.
The attempt is made to complement the following properties: size (small $[AVGSTC]$, medium $[
LPNQDEMIK]$, large $[WFYHR]$), 
hydrophobicity (hydrophobic $[LPVAMWFI]$, polar $[GTCY]$), charge (positive $[KHR]$,
 negative $[DE]$), aromaticity $[WFYH]$, long aliphatic chain $[EKRQM]$, 
 OH-group possession $[SDETY]$, and NH2 group possession $[NQRK]$.
The suggestions are listed according to how different they appear from the original amino acid, 
and they are grouped in round bracket   
if they appear equally good (or rather,equally  destructive) as mutation for the original.
These sugestions are tentative - they might prove disruptive to the fold rather than to the interaction.
