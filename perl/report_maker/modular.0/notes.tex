\section{Notes on using trace results}

\subsection{Coverage} Trace  results are commonly expressed in terms of coverage: the residue is
important if its ``coverage'' is small - that is if it belongs to some small top percentage of residues [100\% is
 all of the residues in a chain], according to trace.
The ET results are presented  in the form of a table, usually limited to top 25\%  percent of residues
(or to some nearby percentage), sorted by the strength of the presumed evolutionary pressure.
(I.e., the smaller the coverage, the stronger the pressure on the residue.) Starting from the top of that list,
mutating a couple of residues should affect the protein somehow, with the exact effects to be determined experimentally.